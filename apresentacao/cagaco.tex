
\section*{Extra}

	\begin{frame}{Projetos Futuros}
		\begin{itemize}
			\item Alterar o funcionamento da comunicação USB para que tenha uma via só para envio de controle e outra só para envio de dados\footnote{Se possível}.
			\item Adicionando o protocolo USB no FPGA:
			\begin{itemize}
				\item Mensagens de informações ao usuário deverão ser trabalhadas no próprio \textit{Driver} eliminando o dispositivo intermediário\footnote{Hipótese.}; ou
				\item Criação de um \textit{software} a nível de usuário para todo o controle do \textit{hardware}.
			\end{itemize}  
		\end{itemize}
	\end{frame}




		\begin{frame}{FPGA -  Implementações}
			\begin{enumerate}
				\item Algoritmo 3DES de criptografia simétrico desenvolvido em hardware onde:
				\begin{itemize}
					\item 3 chaves de 64 bits (o mesmo que 8 caracteres ASCII de 8 bits);
					\item Texto formado em blocos de 64 bits completos.
				\end{itemize}
				\item Algoritmos de envio e recebimento de dados através de UART.
				\begin{itemize}
  					\setlength\itemsep{.8em}
					\item $Clock_{FPGA} = 50.000.000 \quad\quad Clock_{requerido} = 9.600 $
					\item Cada intervalo teria: $\frac{50.000.000}{9.600} = 5.208,33333~$ clocks do $Clock_{FPGA}$
					\item Ou seja, a cada $ 5.208,33333~$ pulso de clock de $50Mhz$, deve pular 1 do $ NovoClock_{FPGA} $.
					%\item Entretanto, é um número \textit{fracionário} criando uma margem de erro que \textit{pode ser} notável assintoticamente.
					\item Erro: $\frac{|Clk_{Experi} - Clk_{Correto}|}{Clk_{Correto}} = \frac{0,333~}{5.208,0} = 0,06400409626\% $
				\end{itemize}
				\item Máquina de estados controladora do sistema em hardware.
			\end{enumerate}
		\end{frame}


		\begin{frame}{Linux - Detalhes importantes do device driver}
			\begin{itemize}
				\item Então, o `novo' driver possui:
				\begin{itemize} 
					\item Macro de inicialização (\texttt{init(), exit(), probe(), disconnect()});
					\item Funções de envio de pacotes;
					\item Funções como \texttt{open()} e \texttt{close()} para a inicialização e términdo da comunicação;
					\item Uma tabela de devices para possíveis conexões.
				\end{itemize}
				\item E deixou de existir:
				\begin{itemize}
					\item \texttt{Init()};
					\item \texttt{Exit()};
					\item \texttt{Probe()};
					\item \texttt{Disconnect()}...
				\end{itemize}
			\end{itemize}
		\end{frame}

	\begin{frame}[fragile, allowframebreaks]{Complementar}
		\begin{comment}
		\begin{block}{Por que o processamento não fica todo no FPGA?}
			Por ser um hardware extremamente rápido, este ficará exercendo sua função de forma \textbf{dedicada}. Ou seja, se determinado usuário quiser trocar o algoritmo na FPGA, basta ele colocar o novo algoritmo desde que ele funcione por meio de blocos.
		\end{block}

		\begin{block}{Por que não desenvolveu um software em espaço de usuário mais bonito ou funcional? Nem sequer falou de espaço do usuário...}
			O foco do desenvolvimento deste trabalho é justamente a conexão do Kernel ao FPGA. Qualquer software em nível de usuário será tido como trabalho extra.
		\end{block}

		\begin{block}{Poderia ter feito num FPGA com tamanho menor?}
			Sim. Tentou-se desenvolver os algoritmos num Ciclone II, entretanto, este só suporta cerca de 4 mil EL. Então utilizou-se o Ciclone IV disponibilizado pelo IFMG - Formiga que possui cerca de 115 mil.
		\end{block}
		\end{comment}

		\begin{block}{Como é o envio de um URB?}
			\begin{enumerate}
				\item No driver, associa um específico endpoint a um específico device.
				\item Envia do driver para o USB Core
				\item Envia do USB Core para um específico driver Host  Controlardor USB.
				\item Este driver Host Controlador processa essa informação e envia para o device.
				\item Quando o envio é completado, o driver Controlador Host notifica o driver do desenvolvedor.
			\end{enumerate}
		\end{block}

		\begin{block}{Major e Minor Numbers}
			\begin{itemize}
				\item Drivers de Char possuem dois números como parte de suas careacterísticas.
				\begin{itemize}
					\item \textbf{Major:} Identifica que o driver está associado com o device.
					\item \textbf{Minor:}
				\end{itemize}
				\item Por exempo o \texttt{/dev/null} e \texttt{/dev/zero}:
				\begin{itemize}
					\item Ambos tem o número Major igual;
					\item Ele é utilizado para associar o driver ao dispositivo;
					\item Entretanto, possuim o número Minor diferentes.
					\item É usado pelo quernel para determinar qual device específico ele está referenciando.
					\item ...
				\end{itemize}
			
			\end{itemize}
		\end{block}

		\begin{block}{Device}
			\begin{itemize}
				\item Endpoints:
				\begin{itemize}
				    \item Control, Interrupt, Bulk e Isochronous.
				\end{itemize}
				\item Interfaces:
				\begin{itemize}
				    \item Mouse, Teclado possuem somente 1 via;
				    \item Podem ter configurações diferentes;
				    \item Áudio em Linux, por exemplo teria 2 interfaces (Áudio e Controle).
				\end{itemize}
				\item Configurations:
				\begin{itemize}
				    \item Em Linux, não é comum utilizar mais que 1 configuração no dispositivo.
				\end{itemize}
				\item Ou seja:
				\begin{enumerate}
				    \item Devices usually have one or more configurations.
				    \item Configurations often have one or more interfaces.
				    \item Interfaces usually have one or more settings.
				    \item Interfaces have zero or more endpoints.
				\end{enumerate}
			\end{itemize}
		\end{block}


		\begin{comment}
		\begin{block}{Compilação e Execução} 
			\begin{itemize}
				\item Deve compilar o código utilizando um compilador fornecido e integrado no Kernel:
				\begin{itemize}
					\item
				\end{itemize}
				\item Após a compilação...:
				\begin{itemize}
					\item
				\end{itemize}
			\end{itemize}
		\end{block}
		\end{comment}
\end{frame}










	\subsection*{Linguagens de Programação}
	\begin{frame}[fragile]{Linguagens de Programação - VHDL}
		\begin{itemize}
			\item \textit{\textbf{V}HSIC (Very High Speed Integrated Circuits) \textbf{H}ardware \textbf{D}escription \textbf{L}anguage}.

		\begin{lstlisting}[language=VHDL]
-- importa std_logic da IEEE library
library IEEE;
use IEEE.std_logic_1164.all;

-- Declara uma entidade
entity ANDGATE is
   port ( 
         IN1 : in std_logic;
         IN2 : in std_logic;
         OUT1: out std_logic);
end ANDGATE;
architecture RTL of ANDGATE is

begin
  OUT1 <= IN1 and IN2;
end RTL;
		\end{lstlisting}
		\end{itemize}
\end{frame}
	\begin{frame}[fragile]{Linguagens de Programação - Arduino}
	\begin{itemize}
		\item Linguagem Arduino.
		\begin{lstlisting}[language=C++]
// the setup function runs once when you press reset or power the board
void setup() {
  // initialize digital pin 13 as an output.
  pinMode(13, OUTPUT);
}

// the loop function runs over and over again forever
void loop() {
  digitalWrite(13, HIGH);  // turn the LED on (HIGH is the voltage level)
  delay(1000);             // wait for a second
  digitalWrite(13, LOW);   // turn the LED off by making the voltage LOW
  delay(1000);             // wait for a second
}			
		\end{lstlisting} 
	\end{itemize}
\end{frame}

	\begin{frame}[fragile]{Compilação do Driver}
			\begin{lstlisting}
obj-m	:= usbSerial.o

KERNELDIR ?= /lib/modules/$(shell uname -r)/build
PWD       := $(shell pwd)

all:
	rm -f *.o *~ core .depend .*.cmd *.ko *.mod.c *.symvers *.order     
	#sudo /usr/sbin/ntpdate ntp.on.br time.nist.gov pool.ntp.org

	# Compilando os arquivos atraves do kernel
	$(MAKE) -C $(KERNELDIR) M=$(PWD)     
	
	# Copiando para a pasta temporaria
	sudo cp usbSerial.ko /tmp/        
	
	sudo rmmod usbSerial             # Removendo o driver antigo
	
	sudo insmod /tmp/usbSerial.ko    # Adicionando o novo driver
			\end{lstlisting}
\end{frame}

